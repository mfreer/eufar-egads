%% $Date: 2012-07-06 17:42:54#$
%% $Revision$
\index{temp\_potential\_equiv\_cnrm}
\algdesc{Equivalent Potential Temperature}
{ %%%%%% Algorithm name %%%%%%
temp\_potential\_equiv\_cnrm
}
{ %%%%%% Algorithm summary %%%%%%
Calculates equivalent potential temperature of air. The equivalent potential temperature is the temperature a parcel of air would reach
if all water vapor in the parcel condensed, and the parcel was brought adiabatially to 1000 hPa.
}
{ %%%%%% Category %%%%%%
Thermodynamics
}
{ %%%%%% Inputs %%%%%%
$T_s$ & Vector & Static temperature [K or $\circ$C] \\
$\theta$ & Vector & Potential temperature [K or $\circ$C] \\
$r$ & Vector & Vater vapor mixing ratio [g/kg] \\
$c_{pa}$ & Coeff. & Specific heat of dry air at constant pressure
}
{ %%%%%% Outputs %%%%%%
$\theta_e$ & Vector & Equivalent potential temperature [same units as $T_s$]
}
{ %%%%%% Formula %%%%%%
\begin{displaymath}
 \theta_e = \theta \left(1 + r \frac{L}{c_{pa} T_s} \right)
\end{displaymath}
%
where $L = 3136.17 - 2.34 T_s$ (for $T_s$ in K)
}
{ %%%%%% Author %%%%%%
CNRM/GMEI/TRAMM
}
{ %%%%%% References %%%%%% 
From the CAM routine which is identical to the algorithm P.~Durand cited in the formula book created for PYREX.
}


