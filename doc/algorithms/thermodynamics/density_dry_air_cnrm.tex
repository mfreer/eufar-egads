%% $Date$
%% $Revision$

\algdesc{Density of dry air}
{ %%%%%% Algorithm name %%%%%%
density\_dry\_air\_cnrm
}
{ %%%%%% Algorithm summary %%%%%%
Calculates density of dry air given static temperature and pressure.
}
{ %%%%%% Category %%%%%%
Thermodynamics
}
{ %%%%%% Inputs %%%%%%
$P_s$ & Vector & Static pressure [hPa] \\
$T_s$ & Vector & Static temperature [K or \deg C]
}
{ %%%%%% Outputs %%%%%%
$\rho$ & Vector & Density of dry air [kg/m$^3$]
}
{ %%%%%% Formula %%%%%%
\begin{displaymath}
 \rho = \frac{100 P_s}{R_a T_s}
\end{displaymath}
%
with $R_a = 287.05$ J kg$^-1$ K$^-1$ 

Density of humid air can be calculated using this same algorithm by using virtual temperature instead of static temperature.
}
{ %%%%%% Author %%%%%%
CNRM/GMEI/TRAMM
}
{ %%%%%% References %%%%%% 
Equation d'\'etat d'un gaz parfait, Triplet-Roche \cite{Triplet}, page 34.
}


