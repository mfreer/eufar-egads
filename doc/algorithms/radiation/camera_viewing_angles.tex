%% $Date: 2012-07-06 17:42:54#$
%% $Revision$
\index{camera\_viewing\_angles}
\algdesc{Camera Viewing Angles}
{ %%%%%% Algorithm name %%%%%%
camera\_viewing\_angles
}
{ %%%%%% Algorithm summary %%%%%%
Calculates per-pixel camera viewing angles of a digital camera given its sensor dimension and
focal length. x--y coordinates are defined as having the left side of the image (x=0) aligned with the flight direction and y=0 to the top of the image.
}
{ %%%%%% Category %%%%%%
Radiation
}
{ %%%%%% Inputs %%%%%%
$n_x$ & Coeff & Number of pixels in x direction \\
$n_y$ & Coeff & Number of pixels in y direction \\
$l_x$ & Coeff & Length of the camera sensor in x direction [mm] \\
$l_y$ & Coeff & Length of the cameras sensor in y direction [mm] \\
$f$ & Coeff & Focal length of the camera lens [mm] \\
}
{ %%%%%% Outputs %%%%%%
$\theta_c$ & Array[$n_x$,$n_y$] & Camera viewing zenith angle [deg] \\
$\Phi_c$ & Array[$n_x$,$n_y$] & Camera viewing azimuth angle [deg], mathematic negative system with 0\deg into flight direction, clockwise\\
}
{ %%%%%% Formula %%%%%%

For each $i$, $j$  where $0<i<n_x$ and $0<j<n_y$:
%
\begin{gather*}
x = l_x \frac{(i - n_x/2)}{n_x} \\
y = l_y \frac{(i - n_y/2)}{n_y} \\
d = \sqrt{x^2 + y^2} \\
\\
\theta_c(i,j) = 2 \tan^{-1}{\frac{d}{2 f}} \\
\Phi_c(i,j) = 2 \pi - \tan^{-1}{\frac{y}{x}}
\end{gather*}
}
{ %%%%%% Author %%%%%%
Andre Ehrlich, Leipzig Institute for Meteorology (a.ehrlich@uni-leipzig.de)
}
{ %%%%%% References %%%%%% 

}


