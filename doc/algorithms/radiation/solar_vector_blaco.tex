%% $Date: 2012-01-31 15:05:08 +0100 (Tue, 31 Jan 2012) $
%% $Revision: 107 $
\index{solar\_vector\_blanco}
\algdesc{Solar Vector Computation}
{ %%%%%% Algorithm name %%%%%%
solar\_vector\_blanco
}
{ %%%%%% Algorithm summary %%%%%%
This algorithm comuptes the current solar vector, given current date, time, latitude and longitude. 
}
{ %%%%%% Category %%%%%%
Radiation
}
{ %%%%%% Inputs %%%%%%
$Symbol$ & Vector/Coeff & Full name [units] \\
}
{ %%%%%% Outputs %%%%%%
$Symbol$ & Vector/Coeff & Full name [units] \\
}
{ %%%%%% Formula %%%%%%
\begin{displaymath}
jd = \frac{1461 (y + 4800 + (m - 14)/12)}{4} + \frac{367 (m - 2 - 12 ((m - 14)/12))}{12} -
     \frac{3 (y + 4900 + (m - 14) / 12) /100 }{4} + d - 32075 - 0.5 + hour/24.0
\end{displaymath}
where $y$ is the year, $m$ is the month, $d$ is the day of the month and $hour$ is the current hour
in decimal format, i.e. with minutes and seconds as fractions of an hour.

The ecliptic coordinates of the sun are computed from the Julian Day by:
\begin{displaymath}
 n = jd - 2451545.0

\Omega = 2.1429 - 0.0010394594 n

L = 4.8950630 + 0.017202791698 n

g = 6.2400600 + 0.0172019699 n

l = L + 0.03341607 \sin{g} + 0.00034894 \sin{2 g} - 0.0001134 - 0.0000203 \sin{\Omega}

ep = 0.4090928 - 6.2140E^{-9} n + 0.0000396 \cos{\Omega}
\end{displaymath}

The conversion from ecliptic coordinates to celestial coordinates is computed by:
\begin{displaymath}
 ra = \tan^{-1} \left[ \frac{\cos{ep} \sin{l}}{\cos{l}} \right]

\delta = \sin^{-1} [\sin{ep} \sin{l}]
\end{displaymath}
where $ra$ must be between 0 and 2$\pi$. 

The conversion between celestial coordinates to horizontal coordinates is then computed by the 
following equations:
\begin{displaymath}
 gmst = 6.6974243242 + 0.0657098283 n + hour

lmst = \frac{pi}{180} (15 gmst + long)

\omega = lmst - ra

\theta_z = \cos^{-1} [ \cos{\Phi} \cos{\omega} \cos{\delta} + \sin{\delta} \sin{\Phi}]

\gamma = \tan^{-1} \left[ \frac{ - \sin{\omega}}{\tan{\delta} \cos{\Phi} - \sin{\Phi} \cos{\omega}} \right]

Parallax = \frac{EarthMeanRadius}{AstronomicalUnit} \sin{\theta_z}

\theta_z = \theta_z + Parallax
\end{displaymath}

where: $EarthMeanRadius$ = 6371.01 km and $AstronomicalUnit$ = 149597890 km
 
}
{ %%%%%% Author %%%%%%

}
{ %%%%%% References %%%%%% 
Manuel Blanco-Muriel, et al., ``Computing the Solar Vector,'' \emph{Solar Energy} 70 (2001): 436-38.
\cite{Blanco} 
}


