%% $Revision$
\index{DGVI2}
\algdescgeneral{DGVI2}
%
{ %%%%%% Algorithm name %%%%%%
biophys\_indices (DGVI2 is one index calculated within the overall program)
}
%
{ %%%%%% Algorithm summary %%%%%%
Calculation of Derivative-based Green Vegetation Index (DGVI). Surface under curve of second derivative between 626nm and 795nm.
}
%
{ %%%%%% Category %%%%%%
Biophysics - red edge parametrisation
}
%
{ %%%%%% Inputs %%%%%%
Narrow band multi- or hyperspectral imagery (ENVI standard image data) including channels close to the wavelengths of 626nm and 795nm.\bigskip
}
%
{ %%%%%% Outputs %%%%%%
Single band with DGVI2 values
}
%
{ %%%%%% Formula %%%%%%
\begin{displaymath}
DGVI2 = \int _{\lambda_1 =626nm}^{\lambda_2 =795nm} \mid \frac{d \rho}{d^2  \lambda }\mid d \lambda
\end{displaymath}
}
%
{ %%%%%% Author %%%%%%
DLR-DFD
}
%
{ %%%%%% References %%%%%% 
Elvidge, C.D., Chen, Z.(1995): Comparison of Broad-Band and Narrow-Band Red and Near-Infrared Vegetation indices. In: Remote Sensing of Environment, 54, p.38-48.
}
