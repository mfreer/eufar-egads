%% $Date: 2012-07-06 17:42:54#$
%% $Revision$
\index{SWIRVI}
\algdescgeneral{SWIRVI}
%
{ %%%%%% Algorithm name %%%%%%
biophys\_indices (SWIRVI is one index calculated within the overall program)
}
%
{ %%%%%% Algorithm summary %%%%%%
Calculation SWIR index: green (SWIRVI)
}
%
{ %%%%%% Category %%%%%%
Biophysics - cover indices
}
%
{ %%%%%% Inputs %%%%%%
Narrow band multi- or hyperspectral imagery (ENVI standard image data) including channels close to the wavelengths of 2090nm, 2210nm and 2280nm.\bigskip
}
%
{ %%%%%% Outputs %%%%%%
Single band with SWIRVI values
}
%
{ %%%%%% Formula %%%%%%
\begin{displaymath}
SWIRVI = 37.72 * (R_{2210}-R_{2090}) +26.27 * (R_{2280} - R_{2090}) +0.57
\end{displaymath}
}
%
{ %%%%%% Author %%%%%%
DLR-DFD
}
%
{ %%%%%% References %%%%%% 
Lobell, D.B., Asner, G.P., Law, B.E., Treuhaft R.N. (2001): Subpixel canopy cover estimation of coniferous forests in Oregon using SWIR imaging spectrometry. In: Journal of geophysical research, 106, p.5151-5160
}
