%% $Date: 2010-09-17 17:53:24 +0200 (Fri, 17 Sep 2010) $
%% $Revision: 24 $
\index{sample\_area\_oap\_all\_in\_raf}
\algdesc{Sample area for imaging probes}
{ %%%%%% Algorithm name %%%%%%
sample\_area\_oap\_all\_in\_raf
}
{ %%%%%% Algorithm summary %%%%%%
Calculation of 'all in' sample area size for OAP probes such as the 2DC, 2DP, CIP, etc. This sample area varies by
number of shadowed diodes. This routine calculates a sample area per bin.
}
{ %%%%%% Category %%%%%%
Microphysics
}
{ %%%%%% Inputs %%%%%%
R & Vector & Particle radius [$\mu$m] \\
$\lambda$ & Coeff. & Laser wavelength [$\mu$m] \\
$D_{arms}$ & Coeff. & Distance between probe arm tips [mm] \\
dL & Coeff. & Diode diameter [$\mu$m] \\
M & Coeff. & Probe magnification factor \\
N & Coeff. & Number of diodes in array
}
{ %%%%%% Outputs %%%%%%
SA & Vector & Sample area [m$^2$]
}
{ %%%%%% Formula %%%%%%
\begin{displaymath}
DOF_i = \frac{6 R_i^2}{\lambda}
\end{displaymath}
where $DOF$ must be less than $D_{arms}$. The parameter $i$ ranges from 1 to $N$.

\begin{displaymath}
ESW_i = \frac{dL(N-X_i-1)}{M}
\end{displaymath}
A value for $ESW_i$ (effective sample width) is calculated for each $X$, where $X$ ranges from 1 to $N$.

\begin{displaymath}
 SA_i = (DOF_i)(ESW_i) 
\end{displaymath}
}
{ %%%%%% Author %%%%%%
NCAR-RAF
}
{ %%%%%% References %%%%%% 
NCAR-RAF Bulletin No. 24 -- \href{http://www.eol.ucar.edu/raf/Bulletins/bulletin24.html}{http://www.eol.ucar.edu/raf/Bulletins/bulletin24.html}
}

