%
% Important note: This latex document should be generated by pdflatex
% to get pdf support right.
%


%%%%%%%%%%%%%%%%%%%%%%%%%%%%%%%%%%%%%%%%%%%%%%%%%%%%%%%%%%%%%%%%%%%%%%%%%%%%%%%%%%%%%%%%%%%%%%%%%%
%%%%%%%%%%% Document class and package options
%%%%%%%%%%%%%%%%%%%%%%%%%%%%%%%%%%%%%%%%%%%%%%%%%%%%%%%%%%%%%%%%%%%%%%%%%%%%%%%%%%%%%%%%%%%%%%%%%%
\documentclass[a4paper,11pt]{report}
\usepackage{graphicx}
\usepackage{hyperref}
\usepackage{fancyvrb}
\usepackage{paralist}
\usepackage{makeidx}
\usepackage{color,calc}
\definecolor{shade}{gray}{0.8}
\newcommand{\makeshadebox}[1]{\fcolorbox{black}{shade}{\makebox{#1}}}

\newcommand{\egads}{\textsc{egads} }
\newcommand{\Egads}{\textsc{Egads} }

\newcounter{actioncounter}
\addtocounter{actioncounter}{1}

\newcommand{\mf}{M\'et\'eoFrance}

\newcommand{\algdesc}[8]{
\section{#1}\bigskip

\textbf{Algorithm name:} #2 \bigskip

\textbf{Summary:} #3 \bigskip

\begin{tabular}{p{2cm} p{2cm} p{8cm}}
	\multicolumn{3}{l}{\textbf{Inputs:}}\\
	\hline
	#4
\end{tabular}

\vspace{0.5cm}


\begin{tabular}{p{2cm} p{2cm} p{8cm}}
	\multicolumn{3}{l}{\textbf{Outputs:}}\\
	\hline
	#5
\end{tabular}

\vspace{1cm}

\textbf{Formula:} #6
\bigskip

\textbf{Source:} #7 \bigskip

\textbf{References:}  #8

\pagebreak

}




%%%%%%%%%%%%%%%%%%%%%%%%%%%%%%%%%%%%%%%%%%%%%%%%%%%%%%%%%%%%%%%%%%%%%%%%%%%%%%%%%%%%%%%%%%%%%%%%%%
%%%%%%%%%%% Include page set-up options
%%%%%%%%%%%%%%%%%%%%%%%%%%%%%%%%%%%%%%%%%%%%%%%%%%%%%%%%%%%%%%%%%%%%%%%%%%%%%%%%%%%%%%%%%%%%%%%%%%
\input{pagesetup.cfg}
\makeindex
\begin{document}
\renewcommand{\thepage}{\roman{page}}

%%%%%%%%%%%%%%%%%%%%%%%%%%%%%%%%%%%%%%%%%%%%%%%%%%%%%%%%%%%%%%%%%%%%%%%%%%%%%%%%%%%%%%%%%%%%%%%%%%
%%%%%%%%%%% Title Page
%%%%%%%%%%%%%%%%%%%%%%%%%%%%%%%%%%%%%%%%%%%%%%%%%%%%%%%%%%%%%%%%%%%%%%%%%%%%%%%%%%%%%%%%%%%%%%%%%%
\include{cover/doc_cover}




%%%%%%%%%%%%%%%%%%%%%%%%%%%%%%%%%%%%%%%%%%%%%%%%%%%%%%%%%%%%%%%%%%%%%%%%%%%%%%%%%%%%%%%%%%%%%%%%%%
%%%%%%%%%%% Indices
%%%%%%%%%%%%%%%%%%%%%%%%%%%%%%%%%%%%%%%%%%%%%%%%%%%%%%%%%%%%%%%%%%%%%%%%%%%%%%%%%%%%%%%%%%%%%%%%%%
\tableofcontents
\pagebreak


%%%%%%%%%%%%%%%%%%%%%%%%%%%%%%%%%%%%%%%%%%%%%%%%%%%%%%%%%%%%%%%%%%%%%%%%%%%%%%%%%%%%%%%%%%%%%%%%%%
%%%%%%%%%%% Introduction
%%%%%%%%%%%%%%%%%%%%%%%%%%%%%%%%%%%%%%%%%%%%%%%%%%%%%%%%%%%%%%%%%%%%%%%%%%%%%%%%%%%%%%%%%%%%%%%%%%
\renewcommand{\thepage}{\arabic{page}}
\setcounter{page}{1}

\chapter{Introduction}

The motivation behind \egads (\textsc{Eufar} General Airborne Data-processing Software) was to create a general method for processing airborne atmospheric sciences data from all...

The current \egads implementation allows users to interact with data processing algorithms using the Python command-line or via Python scripts. Python was chosen due to its straightforward syntax and flexibility between multiple systems. Central to \egads is a data type that encapsulates data and metadata into one object. This simplifies the housekeeping of such data, and allows it to be easily passed between algorithms and data files.


%%%%%%%%%%%%%%%%%%%%%%%%%%%%%%%%%%%%%%%%%%%%%%%%%%%%%%%%%%%%%%%%%%%%%%%%%%%%%%%%%%%%%%%%%%%%%%%%%%
%%%%%%%%%%% Chapters
%%%%%%%%%%%%%%%%%%%%%%%%%%%%%%%%%%%%%%%%%%%%%%%%%%%%%%%%%%%%%%%%%%%%%%%%%%%%%%%%%%%%%%%%%%%%%%%%%%

\chapter{Installation}

The latest version of \egads can be obtained from http://eufar-egads.googlecode.com

\section{Prerequisites}

Building \egads requires the following software to be installed:

\begin{description}
 \item[Python] 2.5 or newer. Available at \href{http://www.python.org/}{http://www.python.org/}.
 \item[numpy] 1.3.0 or newer. Available at \href{http://numpy.scipy.org/}{http://numpy.scipy.org/}.
 \item[scipy] 0.6.0 or newer. Available at \href{http://www.scipy.org/}{http://www.scipy.org/}.
 \item[Python netCDF4 libraries] 0.8.2. Available at \href{http://code.google.com/p/netcdf4-python/}{http://code.google.com/p/netcdf4-python/}.
\end{description}

\section{Optional Packages}

The following are useful when using or compiling \egads:

\begin{description}
 \item[IPython] An optional package which simplifies Python command line usage (\href{http://ipython.scipy.org}{http://ipython.scipy.org}). IPython is an enhanced interactive Python shell which supports tab-completion, debugging, history, etc. 

 \end{description}

\section{Installation}
To install \egads, type \verb|python setup.py install| from the command line. To install to a user-specified location, type \verb|python setup.py install --prefix=$MYDIR|. 

\section{Testing}

To test \egads after it is installed, run the following commands in Python:

\begin{verbatim}
   >>> import egads
   >>> egads.test()
\end{verbatim} 

\chapter{Tutorial}

\section{Command-line usage}
The simplest way to start working with \egads is to run it from the Python command line. To load egads into the Python namespace, simply import it:

\begin{verbatim}
    >>> import egads
\end{verbatim}
You may then begin working with any of the algorithms contained in \egads.

\subsection{Exploring \egads}
There are several useful methods to explore routines contained in \egads. The first is using the Python built-in \verb|dir()| command:

\begin{verbatim}
    >>> dir(egads)
\end{verbatim}
returns all the classes and subpackages contained in \egads. \Egads follows the naming conventions from the Python Style Guide (\href{http://www.python.org/dev/peps/pep-0008}{http://www.python.org/dev/peps/pep-0008}), so classes are always \verb|MixedCase|, functions and modules are generally \verb|lowercase| or \verb|lowercase_with_underscores|. As a further example,

\begin{verbatim}
    >>> dir(egads.input)
\end{verbatim}
would returns all the classes and subpackages of the \verb|egads.input| module.

Another way to explore \egads is by using tab completion, if supported by your Python installation. Typing 

\begin{verbatim}
    >>> egads.
\end{verbatim}
then hitting \textsc{tab} will return a list of all available options. 

Python has built-in methods to display documentation on any function known as docstrings. The easiest way to access them is using the \verb|help()| function:

\begin{verbatim}
   >>> help(egads.input.NetCdf)
\end{verbatim}
will return all methods and their associated documentation for the NetCdf class.

\subsection{Working with NetCDF files}

\Egads provides two classes to work with NetCDF files. The simplest, \verb|egads.input.NetCdf()|, allows simple read/write operations to NetCDF files. The other, \verb|egads.input.EgadsNetCdf()|, is designed to interface with NetCDF files conforming to the N6SP data and metadata regulations. This class directly reads NetCDF data into an \verb|EgadsData| class.

\subsubsection{Opening}

To open a NetCDF file, simply create a \verb|egads.input.NetCdf()| instance and then use the \verb|open(pathname, permissions)| command:

\begin{verbatim}
    >>> import egads
    >>> f = egads.input.NetCdf()
    >>> f.open('/pathname/filename','r')
\end{verbatim}

Valid values for permissions are:

\begin{description}
 \item[r] Read: opens file for reading only. Default value if nothing is provided.
 \item[w] Write: opens file for writing, and overwrites data in file.
 \item[a] Append: opens file for appending data.
 \item[r+] Same as 'a'.
\end{description}


\subsubsection{Getting info}



\subsubsection{Reading data}

\subsubsection{Writing data}

\subsubsection{Closing}





%%%%%%%%%%%%%%%%%%%%%%%%%%%%%%%%%%%%%%%%%%%%%%%%%%%%%%%%%%%%%%%%%%%%%%%%%%%%%%%%%%%%%%%%%%%%%%%%%%
%%%%%%%%%%% APPENDIX inclusions
%%%%%%%%%%%%%%%%%%%%%%%%%%%%%%%%%%%%%%%%%%%%%%%%%%%%%%%%%%%%%%%%%%%%%%%%%%%%%%%%%%%%%%%%%%%%%%%%%%
\begin{appendix}

\chapter{Class reference}

 
\end{appendix}







%%%%%%%%%%%%%%%%%%%%%%%%%%%%%%%%%%%%%%%%%%%%%%%%%%%%%%%%%%%%%%%%%%%%%%%%%%%%%%%%%%%%%%%%%%%%%%%%%%
%%%%%%%%%%% Bibliography
%%%%%%%%%%%%%%%%%%%%%%%%%%%%%%%%%%%%%%%%%%%%%%%%%%%%%%%%%%%%%%%%%%%%%%%%%%%%%%%%%%%%%%%%%%%%%%%%%%
%\newpage\addcontentsline{toc}{section}{Bibliography and references}
%\begin{thebibliography}{5}
%\bibitem{Vahdat et al., 2002} Vahdat A., Yocum K., Walsh K., Mahadevan P., Kosti\c{c} D., Chase J., and Becker D. (2002). Scalability and Accuracy in a Large-Scale Network Emulator. \emph{Proceedings of 5th Symposium on Operating Systems Design and Implementation (OSDI)}
%\bibitem{Modelnet} Modelnet. http://issg.cs.duke.edu/modelnet.html.
%\bibitem{ModelNetRelease} Modelnet release page. http://sysnet.ecsd.edu/modelnet/realease.html.
%\bibitem{ModelNetHowto} Modelnet Howto. http://sysnet.ucsd.edu/modelnet/howto.html.
%\bibitem{Holenstein, 2005} Holenstein R. (2005). Scalability and Accuracy in a Large-Scale Network Emulator - CPSC 538A - Paper Presentation
%\bibitem{Harpin, 2001} Harpin, T. (2001). Using java.lang.reflect.Proxy to Interpose on Java Class Methods. \emph{Sun Developer Network (SDN)}. http://java.sun.com/developer/technicalArticles/JavaLP/Interposing/.
%\bibitem{Kostic, 2006} Kostic D. (2005). Modelnet notes. http://resolute.ucsd.edu/dokuwiki/notes:modelnet.
%\end{thebibliography}


\end{document}
%%%%%%%%%%%%%%%%%%%%%%%%%%%%%%%%%%%%%%%%%%%%%%%%%%%%%%%%%%%%%%%%%%%%%%%%%%%%%%%%%%%%%%%%%%%%%%%%%%
%%%%%%%%%%% END OF FILE
%%%%%%%%%%%%%%%%%%%%%%%%%%%%%%%%%%%%%%%%%%%%%%%%%%%%%%%%%%%%%%%%%%%%%%%%%%%%%%%%%%%%%%%%%%%%%%%%%%
