
% 
% \section{Data Formats}
% 
% At the moment in the US there is no universal standard data format.  There are several existing protocols and conventions, however, they are not adopted universally among the various agencies.  In this section, the protocols and conventions for data formats will be outlined.  An attempt to show which agency is using each standard will also be made, however the list may not be inclusive, since demand for and use of these standards is often internal only.
% 
% \subsection{Format Overview}
% 
% There are two data format standards in wide use US operators - NetCDF and NASA Ames.  Other formats in use are either rare in airborne science (e.g., HDF) or have no well defined standards (e.g., CSV).  
% 
% \subsubsection{NetCDF}\label{netcdf}
% %
% NetCDF, developed by Unidata, at UCAR, is a binary format, with the advantages of being self-describing, architecture independent, multi-rate data compatible, and having pre-existing interfaces for many programming languages.  It is extremely well suited to storing variables with different dimension sizes and amounts.  Data within the file can be accessed directly, that is, unlike ASCII, specific variables, dimensions or time frames can be extracted without having to parse the entire file.  
% 
% NetCDF has a wide userbase in the atmospheric and oceanic community - users include NCAR, NASA, NOAA, University of Wyoming, University of South Dakota, and BADC, among others.  There are existing data and metadata standards endorsed by Unidata.  The conventions most applicable to airborne data are the \href{http://cf-pcmdi.llnl.gov/}{Climate and Forecast (CF) conventions} and the \href{http://www.eol.ucar.edu/raf/Software/netCDF.html}{NCAR RAF conventions}, which are built upon the CF conventions.  \href{http://badc.nerc.ac.uk/help/formats/}{BADC} supports and strongly recommends the CF conventions.  A list of conventions for NetCDF can be found below:
% 
% \begin{itemize}
% 	\item Unidata NetCDF Best Practices:
% 
%  \href{http://www.unidata.ucar.edu/software/netcdf/docs/BestPractices.html}{http://www.unidata.ucar.edu/software/netcdf/docs/BestPractices.html}
% 	\item Climate and Forecast (CF) conventions: \href{http://cf-pcmdi.llnl.gov/}{http://cf-pcmdi.llnl.gov/}
% 	\item NCAR RAF conventions: \href{http://www.eol.ucar.edu/raf/Software/netCDF.html}{http://www.eol.ucar.edu/raf/Software/netCDF.html}
% \end{itemize}
% 
% \subsubsection{NASA Ames}\label{nasaames}
% %
% The NASA Ames data format was developed by NASA in the late 1980s and was created to emphasize human readability, portability and data self-description.  It is essentially an ASCII tab delimeted format with well-defined headers and data structure.  A complete description of this format can be found at the \href{http://badc.nerc.ac.uk/help/formats/NASA-Ames/}{BADC's NASA Ames format page http://badc.nerc.ac.uk/help/formats/NASA-Ames/}.  
% 
% 


\section{Data Exchange}\label{dataex}

Real-time data exchange is the sharing of aircraft housekeeping data (nav, state, etc) and probe data in real-time to operators and instruments onboard the aircraft.  With the adoption of Ethernet-enabled probes and data systems, it is now more feasible to broadcast aircraft state and probe data to all networked operators and instruments aboard the aircraft.  Until recently, there were no established standards for real-time data exchange within aircraft.  Operators and scientists were provided data in an ad hoc, need-based manner, with parameters and formats potentially different between instruments, operators and experiments.  There is now an effort underway to develop common real-time housekeeping and probe data transmission formats in the US being led by IWGADTS.  Additionally, there are several proposed XML standards by OGC and Unidata for the transmission of this data.  These items cover a large part of the standards now in use by the major US operators, and are summarized in this section.  

\subsection{IWGADTS}\label{iwgadts}
%
The Interagency Working Group for Airborne Data and Telemetry Systems (IWGADTS, see Appendix \ref{iwgadtsdesc}) is a collaborative structure set up between the major airborne research aircraft operators in the US to develop recommendations for increased interoperability between aircraft platforms and payloads.  Members include NCAR, NASA, NOAA, NSF, DOE, DOI, and ONR.  IWGADTS has developed recommendations for real-time data exchange within aircraft, and these recommendations are being widely adopted within the US airborne research fleets.

\subsubsection{IWG1}
The Interagency Working Group standard format number 1 (IWG1) is a 1 Hz, ASCII CSV packet containing a preset list of aircraft state and housekeeping parameters including latitude, longitude, true airspeed, roll, pitch, etc.  The purpose of this packet is to provide a standard list of variables which will be available during all research flights for common use.  IWG1 is extendable, if desired, by the addition of supplementary parameters at the end of the packet.  The packet structure is straightforward - comma separated values starting with `IWG1,' followed by an ISO-8601 timestamp, followed by the 21 predetermined variables (see Appendix \ref{paramlist}), with unavailable values left blank.  The following is an example of the format:

	\begin{verbatim}
		IWG1,yyyymmddThhmmss,value,value,,value,...,value\r\n
	\end{verbatim}


A full description of the IWG1 packet definition can be found at the following website -- \href{http://www.eol.ucar.edu/iwgadts/products/real-time-data-feed}{http://www.eol.ucar.edu/iwgadts/products/real-time-data-feed} and in Appendix \ref{iwg1}.

IWGADTS does not define the implementation of the feed (UDP vs RS-232).  The IWG1 packet can be used on either system.

The IWG1 packet is being widely adopted in the US by agencies such as NCAR, NASA, and NOAA.  


\subsection{CSV Packet}\label{csvpkt}
%
IWGADTS also defines an expanded CSV packet for sharing real-time instrument specific data around the aircraft.  This packet has a similar format to the IWG1 packet, but the identifier is changed to the probe-specific identifier, and the list of parameters reflects the probe output.  CSV packets may be transmitted at higher rates than 1 Hz if desired, depending on the available bandwidth.  An example is shown below:

	\begin{verbatim}
		 IDENTIFIER,yyyy-mm-ddThh:mm:ss,value,value,value,,value
		 NOAA_SP2,20090120T145531,15.7738,-96.2707,137.462,6.9332
	\end{verbatim}


The data contained in the CSV packet will vary depending on the instrument.  Hence, the CSV packet requires a metadata header which is sent separately.  The header format is still being determined, but it will likely follow one of the XML formats discussed in the next section.  

A complete description of the CSV packet can be found at \href{http://www.eol.ucar.edu/iwgadts/products/csv-data-packets}{http://www.eol.ucar.edu/iwgadts/products/csv-data-packets} and in Appendix \ref{csv}.


\subsection{XML Descriptor}
%
Extensible Markup Language (XML) is a set of rules for encoding documents in a well-formed, application-readable format.  XML descriptors are useful for describing the metadata for a given packet or instrument type and for the transfer or streaming of probe data.  There are several XML formats in use or proposed for use in the US airborne community.

\subsubsection{NcML}
%
The NetCDF Markup Language (NcML) is an XML format created by Unidata which replicates the header and metadata information found in a NetCDF file.  The specifications for this format can be found at the Unidata website - \href{http://www.unidata.ucar.edu/software/netcdf/ncml/}{http://www.unidata.ucar.edu/software/netcdf/ncml/}.  An example NcML descriptor is given in Appendix \ref{ncml}.  IWGADTS is considering the use of NcML as an XML descriptor for the IWG1 and CSV packets described in Section \ref{iwgadts}.  

\subsubsection{OGC XML Standards}
%
SensorML and TransducerML are XML encodings developed by Open Geospatial Consortium, Inc (OGC).  They are both extremely flexible in describing a wide range of probe and sensor types, and include metadata which describes the methods used to process the sensor data into higher-level products.  Unlike NcML which only describes metadata, these schemes encompass both the data itself and its corresponding metadata.  Descriptions of the two schemes follow. 

From the SensorML website (\href{http://www.opengeospatial.org/standards/sensorml}{http://www.opengeospatial.org/standards/sensorml}):
\begin{quotation}
The OpenGIS Sensor Model Language Encoding Standard (SensorML) specifies models and XML encoding that provide a framework within which the geometric, dynamic, and observational characteristics of sensors and sensor systems can be defined. There are many different sensor types, from simple visual thermometers to complex electron microscopes and earth observing satellites. These can all be supported through the definition of atomic process models and process chains. Within SensorML, all processes and components are encoded as application schema of the Feature model in the Geographic Markup Language (GML) Version 3.1.1. This is one of the OGC Sensor Web Enablement (SWE) [\href{http://www.opengeospatial.org/ogc/markets-technologies/swe}{http://www.opengeospatial.org/ogc/markets-technologies/swe}] suite of standards. For additional information on SensorML, see \href{http://vast.uah.edu/SensorML}{http://vast.uah.edu/SensorML}.

\end{quotation}

TransducerML is more suited to streaming data and also provides methods to send control commands to sensors, actuators and transmitters.  From the TransducerML website (\href{http://www.opengeospatial.org/standards/tml}{http://www.opengeospatial.org/standards/tml}):
\begin{quotation}
 The OpenGIS® Transducer Markup Language Encoding Standard (TML) is an application and presentation layer communication protocol for exchanging live streaming or archived data to (i.e. control data) and/or sensor data from any sensor system. A sensor system can be one or more sensors, receivers, actuators, transmitters, and processes. A TML client can be capable of handling any TML enabled sensor system without prior knowledge of that system.

The protocol contains descriptions of both the sensor data and the sensor system itself. It is scalable, consistent, unambiguous, and usable with any sensor system incorporating any number sensors and actuators. It supports the precise spatial and temporal alignment of each data element. It also supports the registration, discovery and understanding of sensor systems and data, enabling users to ignore irrelevant data. It can adapt to highly dynamic and distributed environments in distributed net-centric operations.
\end{quotation}

NASA currently uses an internal XML format, but they're planning to move to either SensorML or TransducerML.  

% 
% 
% \subsection{Hardware}
% %
% There has been some movement in the last few years towards standardizing a hardware implementation for real-time data transfer.  In the past, data was supplied to the workstations of specific operators over direct serial connections.  With the adoption of Ethernet onboard aircraft, connections to instruments and workstations has become much simpler.  With Ethernet, it is possible for instrument data to be broadcast for use by all networked operators and instruments.
% 
% NASA and NCAR have moved to Ethernet as their primary means of distributing data between instruments and operators on board the aircraft.  Both agencies are recommending that all new probes be equipped with Ethernet interfaces, but will continue to support legacy instruments \textbf{**VERIFICATION NEEDED**}.  NOAA has just implemented a new system which will have ??? (Waiting on NOAA response)
% 

\subsection{Timing}
%
In general, there are three protocols used for time synchronization by operators in the US - direct GPS antenna feed, Network Time Protocol (NTP), and IRIG-B.  The direct GPS antenna feed is generally given at 1 pulse per second, NTP provides single digit millisecond accuracy and IRIG-B provides double-digit microsecond accuracy.  IWGADTS recommends that of these, all aircraft provide at least IRIG-B in UTC.  


\section{Data Links}
%
At the moment, there are no standards in the US for real-time aircraft-to-ground data links.  Many operators have just recently integrated satellite communications into their aircraft, and bandwidth is generally limited.  Additionally, it is expensive to send large amounts of data over such a link.  Thus, the current practice is to customize the data sent to the ground based on the requirements of a specific campaign.  However, in some cases, there is a basic framework for transfer to the ground.  Examples of current satellite data-link setups follow.

NCAR sends its data using an IWGADTS CSV packet compressed using bzip2 at rates varying between one every second to one every five seconds, depending on project requirements.  NCAR has both Iridium\footnote{Transfer rates of 2600 baud. See Appendix \ref{iridium}} and Inmarsat\footnote{Transfer rates up to 128 Kbps.  See Appendix \ref{inmarsat}} satellite modems on its aircraft.  Typical project usage over the Inmarsat includes, chat (IRC), 30 data parameters to ground every other second, satellite imagery upload and small JPEG from aircraft camera every minute.  

NASA is working toward providing Iridium as a standard for low-rate telemetry on all its airborne platforms.  Data transfer is done by port forwarding, and they support arbitrary formats other than CSV.  They will additionally have high-rate satcom capabilities from Inmarsat or K$_{\textrm{u}}$\footnote{Transfer rates up to 50Mbps.}  when feasible.  

NASA's Global Hawk UAV uses four bonded Iridium modems for its low-rate satcom, and can have a K$_{\textrm{u}}$-based high-bandwidth system  installed, depending on project requirements.  For low-rate communications, users are limited to one UDP packet no larger than 550 bytes every 10 seconds.  There are no format requirements for these packets, however, use of compression or binary packet contents is recommended for maximized data throughput.  



 

%Currently, parameters and formats change from project to project, but in some cases, data captured on the ground is converted and can be disseminated in the standard formats discussed above.  






%%%%%%%%%%%%%%%%%%%%%%%%%%%%%%%%%%%%%%%%%%%%%%%%%%%%%%%%%%%%%%%%%%%%%%%%%%%%%%%%%%%%%%%%%%%%%%%%%%
%%%%%%%%%%% END OF FILE
%%%%%%%%%%%%%%%%%%%%%%%%%%%%%%%%%%%%%%%%%%%%%%%%%%%%%%%%%%%%%%%%%%%%%%%%%%%%%%%%%%%%%%%%%%%%%%%%%%