%
% Important note: This latex document should be generated by pdflatex
% to get pdf support right.
%


%%%%%%%%%%%%%%%%%%%%%%%%%%%%%%%%%%%%%%%%%%%%%%%%%%%%%%%%%%%%%%%%%%%%%%%%%%%%%%%%%%%%%%%%%%%%%%%%%%
%%%%%%%%%%% Document class and package options
%%%%%%%%%%%%%%%%%%%%%%%%%%%%%%%%%%%%%%%%%%%%%%%%%%%%%%%%%%%%%%%%%%%%%%%%%%%%%%%%%%%%%%%%%%%%%%%%%%
\documentclass[a4paper,11pt]{report}
\usepackage{graphicx}
\usepackage{hyperref}
\usepackage{fancyvrb}
\usepackage{paralist}
\usepackage{makeidx}
\usepackage{color,calc}
\definecolor{shade}{gray}{0.8}
\newcommand{\makeshadebox}[1]{\fcolorbox{black}{shade}{\makebox{#1}}}


\newcounter{actioncounter}
\addtocounter{actioncounter}{1}

\newcommand{\egads}{\textsc{egads} }
\newcommand{\Egads}{\textsc{Egads} }


\newcommand{\mf}{M\'et\'eoFrance}

\renewcommand{\deg}{$^{\circ}$}

\newcommand{\algdesc}[9]{
\section{#1}\bigskip

\noindent \textbf{Algorithm name:} #2 \bigskip

\noindent \hangindent=2.0cm \textbf{Category:} #4 \bigskip

\noindent \hangindent=2.3cm \textbf{Summary:} #3 \bigskip

\noindent \begin{tabular}{p{2cm} p{2cm} p{8cm}}
	\multicolumn{3}{l}{\textbf{Inputs:}}\\
	\hline
	#5
\end{tabular}

\vspace{0.5cm}


\noindent \begin{tabular}{p{2cm} p{2cm} p{8cm}}
	\multicolumn{3}{l}{\textbf{Outputs:}}\\
	\hline
	#6
\end{tabular}

\vspace{1cm}

\noindent \textbf{Formula:} #7
\bigskip

\noindent \hangindent=1.9cm \textbf{Source:} #8 \bigskip

\noindent \hangindent=2.5cm \textbf{References:}  #9

\pagebreak

}

\newcommand{\algdescgeneral}[9]{
\section{#1}\bigskip

\noindent \textbf{Algorithm name:} #2 \bigskip

\noindent \hangindent=2.0cm \textbf{Category:} #4 \bigskip

\noindent \hangindent=2.3cm \textbf{Summary:} #3 \bigskip

\noindent \textbf{Inputs:} #5 \bigskip

\noindent \textbf{Outputs:} #6 \bigskip

\noindent \textbf{Formula:} #7 \bigskip

\noindent \hangindent=1.9cm \textbf{Source:} #8 \bigskip

\noindent \hangindent=2.5cm \textbf{References:}  #9

\pagebreak

}



%%%%%%%%%%%%%%%%%%%%%%%%%%%%%%%%%%%%%%%%%%%%%%%%%%%%%%%%%%%%%%%%%%%%%%%%%%%%%%%%%%%%%%%%%%%%%%%%%%
%%%%%%%%%%% Include page set-up options
%%%%%%%%%%%%%%%%%%%%%%%%%%%%%%%%%%%%%%%%%%%%%%%%%%%%%%%%%%%%%%%%%%%%%%%%%%%%%%%%%%%%%%%%%%%%%%%%%%
\input{pagesetup.cfg}
\makeindex
\begin{document}
\renewcommand{\thepage}{\roman{page}}

%%%%%%%%%%%%%%%%%%%%%%%%%%%%%%%%%%%%%%%%%%%%%%%%%%%%%%%%%%%%%%%%%%%%%%%%%%%%%%%%%%%%%%%%%%%%%%%%%%
%%%%%%%%%%% Title Page
%%%%%%%%%%%%%%%%%%%%%%%%%%%%%%%%%%%%%%%%%%%%%%%%%%%%%%%%%%%%%%%%%%%%%%%%%%%%%%%%%%%%%%%%%%%%%%%%%%
\include{cover/alg_cover}




%%%%%%%%%%%%%%%%%%%%%%%%%%%%%%%%%%%%%%%%%%%%%%%%%%%%%%%%%%%%%%%%%%%%%%%%%%%%%%%%%%%%%%%%%%%%%%%%%%
%%%%%%%%%%% Indices
%%%%%%%%%%%%%%%%%%%%%%%%%%%%%%%%%%%%%%%%%%%%%%%%%%%%%%%%%%%%%%%%%%%%%%%%%%%%%%%%%%%%%%%%%%%%%%%%%%
\tableofcontents
\pagebreak


%%%%%%%%%%%%%%%%%%%%%%%%%%%%%%%%%%%%%%%%%%%%%%%%%%%%%%%%%%%%%%%%%%%%%%%%%%%%%%%%%%%%%%%%%%%%%%%%%%
%%%%%%%%%%% Introduction
%%%%%%%%%%%%%%%%%%%%%%%%%%%%%%%%%%%%%%%%%%%%%%%%%%%%%%%%%%%%%%%%%%%%%%%%%%%%%%%%%%%%%%%%%%%%%%%%%%
\renewcommand{\thepage}{\arabic{page}}
\setcounter{page}{1}

%%%%%%%%%%%%%%%%%%%%%%%%%%%%%%%%%%%%%%%%%%%%%%%%%%%%%%%%%%%%%%%%%%%%%%%%%%%%%%%%%%%%%%%%%%%%%%%%%%
%%%%%%%%%%% Chapters
%%%%%%%%%%%%%%%%%%%%%%%%%%%%%%%%%%%%%%%%%%%%%%%%%%%%%%%%%%%%%%%%%%%%%%%%%%%%%%%%%%%%%%%%%%%%%%%%%%

\chapter{Introduction}

This document contains descriptions of algorithms contained in the \egads toolbox. Within each algorithm description is the following:

\begin{itemize}
\item \textbf{Algorithm Name} -- name of algorithm as implemented in \egads.
\item \textbf{Category} -- general category of algorithm. Algorithm can be found in this subdirectory in \egads.
\item \textbf{Summary} -- short description of what the algorithm does.
\item \textbf{Inputs} -- expected inputs to algorithm. This field includes expected units, and data type of input.
\item \textbf{Outputs} -- outputs produced by algorithm.
\item \textbf{Formula} -- description of formulas or methods behind the algorithm.
\item \textbf{Source} -- person, institution or entity who provided the algorithm.
\item \textbf{References} -- any references to literature, journals or documents with more information on the current algorithm
\end{itemize}

To aid in algorithm usage, there is a general naming scheme for \egads algorithms. Generally, algorithm names are composed as follows: |{general calculation}_{specific calculation}_{source}|. So, for example, an algorithm provided by CNRM to calculate the density of dry air would be named |density_dry_air_cnrm|.


%\chapter{Spatial Transformations}

%\chapter{Temporal Transformations}

\chapter{Thermodynamics}

%% $Date: 2012-07-06 17:42:54#$
%% $Revision$
\index{altitude\_pressure\_cnrm}
\algdesc{Pressure altitude
}
{ %%%%%% Algorithm name %%%%%%
altitude\_pressure\_cnrm
}
{ %%%%%% Algorithm summary %%%%%%
Calculates pressure altitude using virtual temperature.
}
{ %%%%%% Category %%%%%%
Thermodynamics
}
{ %%%%%% Inputs %%%%%%
$T_v$ & Vector & Virtual temperature [K or \deg C] \\
$P_s$ & Vector & Static pressure [hPa] \\
$P_{surface}$ & Coeff & Surface pressure [hPa] \\ 
$R_a/g$ & Coeff & Gas constant of air over acceleration of gravity
}
{ %%%%%% Outputs %%%%%%
$Alt_p$ & Vector & Pressure altitude [m]
}
{ %%%%%% Formula %%%%%%
\begin{displaymath}
 Alt_p = \frac{R_a}{g} T_v \log\left(\frac{P_{surface}}{P_s}\right)
\end{displaymath}

}
{ %%%%%% Author %%%%%%
CNRM/GMEI/TRAMM
}
{ %%%%%% References %%%%%% 

}




\include{algorithms/thermodynamics/density_dry_air_cnrm}

%% $Date$
%% $Revision$

\algdesc{Relative humidity from capacitive probe
}
{ %%%%%% Algorithm name %%%%%%
hum\_rel\_capacitive\_cnrm
}
{ %%%%%% Algorithm summary %%%%%%
Calculates relative humidity using the measured frequency from a capacitive probe.
}
{ %%%%%% Inputs %%%%%%
$Ucapf$ & Vector & Output frequency of the capacitive probe [Hz]\\ 
$T_s$ & Vector & Static temperature [K] \\
$P_s$ & Vector & Static pressure [hPa] \\
$\Delta P$ & Vector & Dynamic pressure [hPa] \\
$C_t$ & Coeff. & Temperature correction coefficient [\%$^\circ$C] \\
$F_{min}$ & Coeff. & Minimal acceptable frequency [Hz] \\
$C_0$ & Coeff. & 0th degree calibration coefficient \\
$C_1$ & Coeff. & 1st degree calibration coefficient \\
$C_2$ & Coeff. & 2nd degree calibration coefficient 
}
{ %%%%%% Outputs %%%%%%
$H_u$ & Vector & Relative humidity [\%]
}
{ %%%%%% Formula %%%%%%
If $Ucapf \leq F_{min}$ then $Ucapf = F_{min}$

\begin{displaymath}
H_u = \frac{P_s}{P_s + \Delta P} \left[C_0 + C_1 Ucapf + C_2 Ucapf^2 + C_t (T_s-20)\right] \nonumber
\end{displaymath}
with $T_s$ in $^\circ C$.
}
{ %%%%%% Author %%%%%%
CNRM/GMEI/TRAMM
}
{ %%%%%% References %%%%%% 
H.~Bellec and G.~Duverneuil.  Appareils de mesure de l'hygrom\'etrie sur le Merlin IV.  Note de Centre 9, M\'et\'eo-France CNRM/CAM, July 1996.
}

\include{algorithms/thermodynamics/pressure_angle_incidence_cnrm}

\include{algorithms/thermodynamics/temp_potential_cnrm}

%% $Date$
%% $Revision$

\algdesc{Equivalent Potential Temperature}
{ %%%%%% Algorithm name %%%%%%
temp\_potential\_equiv\_cnrm
}
{ %%%%%% Algorithm summary %%%%%%
Calculates equivalent potential temperature of air. The equivalent potential temperature is the temperature a parcel of air would reach
if all water vapor in the parcel condensed, and the parcel was brought adiabatially to 1000 hPa.
}
{ %%%%%% Category %%%%%%
Thermodynamics
}
{ %%%%%% Inputs %%%%%%
$T_s$ & Vector & Static temperature [K or $\circ$C] \\
$\theta$ & Vector & Potential temperature [K or $\circ$C] \\
$r$ & Vector & Vater vapor mixing ratio [g/kg] \\
$c_{pa}$ & Coeff. & Specific heat of dry air at constant pressure
}
{ %%%%%% Outputs %%%%%%
$\theta_e$ & Vector & Equivalent potential temperature [same units as $T_s$]
}
{ %%%%%% Formula %%%%%%
\begin{displaymath}
 \theta_e = \theta \left(1 + r \frac{L}{c_{pa} T_s} \right)
\end{displaymath}
%
where $L = 3136.17 - 2.34 T_s$ (for $T_s$ in K)
}
{ %%%%%% Author %%%%%%
CNRM/GMEI/TRAMM
}
{ %%%%%% References %%%%%% 
From the CAM routine which is identical to the algorithm P.~Durand cited in the formula book created for PYREX.
}




\include{algorithms/thermodynamics/temp_static_cnrm}

\include{algorithms/thermodynamics/temp_virtual_cnrm}

\include{algorithms/thermodynamics/velocity_mach_raf}

\include{algorithms/thermodynamics/velocity_tas_cnrm}

\include{algorithms/thermodynamics/velocity_tas_raf}

\include{algorithms/thermodynamics/velocity_tas_longitudinal_cnrm}







\chapter{Microphysics}

\include{algorithms/microphysics/diameter_mean_raf}

\include{algorithms/microphysics/number_conc_total_raf}

%% $Date: 2010-09-17 17:53:24 +0200 (Fri, 17 Sep 2010) $
%% $Revision: 24 $
\index{sample\_area\_oap\_all\_in\_raf}
\algdesc{Sample area for imaging probes}
{ %%%%%% Algorithm name %%%%%%
sample\_area\_oap\_all\_in\_raf
}
{ %%%%%% Algorithm summary %%%%%%
Calculation of 'all in' sample area size for OAP probes such as the 2DC, 2DP, CIP, etc. This sample area varies by
number of shadowed diodes. This routine calculates a sample area per bin.
}
{ %%%%%% Category %%%%%%
Microphysics
}
{ %%%%%% Inputs %%%%%%
R & Vector & Particle radius [$\mu$m] \\
$\lambda$ & Coeff. & Laser wavelength [$\mu$m] \\
$D_{arms}$ & Coeff. & Distance between probe arm tips [mm] \\
dL & Coeff. & Diode diameter [$\mu$m] \\
M & Coeff. & Probe magnification factor \\
N & Coeff. & Number of diodes in array
}
{ %%%%%% Outputs %%%%%%
SA & Vector & Sample area [m$^2$]
}
{ %%%%%% Formula %%%%%%
\begin{displaymath}
DOF_i = \frac{6 R_i^2}{\lambda}
\end{displaymath}
where $DOF$ must be less than $D_{arms}$. The parameter $i$ ranges from 1 to $N$.

\begin{displaymath}
ESW_i = \frac{dL(N-X_i-1)}{M}
\end{displaymath}
A value for $ESW_i$ (effective sample width) is calculated for each $X$, where $X$ ranges from 1 to $N$.

\begin{displaymath}
 SA_i = (DOF_i)(ESW_i) 
\end{displaymath}
}
{ %%%%%% Author %%%%%%
NCAR-RAF
}
{ %%%%%% References %%%%%% 
NCAR-RAF Bulletin No. 24 -- \href{http://www.eol.ucar.edu/raf/Bulletins/bulletin24.html}{http://www.eol.ucar.edu/raf/Bulletins/bulletin24.html}
}



%% $Date: 2010-09-17 17:53:24 +0200 (Fri, 17 Sep 2010) $
%% $Revision: 24 $
\index{sample\_area\_oap\_center\_in\_raf}
\algdesc{Sample area for imaging probes}
{ %%%%%% Algorithm name %%%%%%
sample\_area\_oap\_center\_in\_raf
}
{ %%%%%% Algorithm summary %%%%%%
Calculation of 'center in' sample area size for OAP probes such as the 2DC, 2DP, CIP, etc. This sample area varies by number of shadowed diodes. This routine is intended to calculate a sample area per bin.
}
{ %%%%%% Category %%%%%%
Microphysics
}
{ %%%%%% Inputs %%%%%%
R & Vector & Particle radius [$\mu$m] \\
$\lambda$ & Coeff. & Laser wavelength [$\mu$m] \\
$D_{arms}$ & Coeff. & Distance between probe arm tips [mm] \\
dL & Coeff. & Diode diameter [$\mu$m] \\
M & Coeff. & Probe magnification factor \\
N & Coeff. & Number of diodes in array \\
}
{ %%%%%% Outputs %%%%%%
SA & Vector & Sample area [m$^2$]
}
{ %%%%%% Formula %%%%%%
\begin{displaymath}
DOF_i = \frac{6 R_i^2}{\lambda}
\end{displaymath}
where $DOF$ must be less than $D_{arms}$. The parameter $i$ ranges from 1 to $N$.

\begin{displaymath}
ESW = N dL \nonumber
\end{displaymath}

\begin{displaymath}
 SA_i = (DOF_i)(ESW) 
\end{displaymath}
}
{ %%%%%% Author %%%%%%
NCAR-RAF
}
{ %%%%%% References %%%%%% 
NCAR-RAF Bulletin No. 24 -- \href{http://www.eol.ucar.edu/raf/Bulletins/bulletin24.html}{http://www.eol.ucar.edu/raf/Bulletins/bulletin24.html}
}



%% $Date: 2010-09-17 17:53:24 +0200 (Fri, 17 Sep 2010) $
%% $Revision: 24 $
\index{sample\_area\_scattering\_raf}
\algdesc{Sample area for scattering probes}
{ %%%%%% Algorithm name %%%%%%
sample\_area\_scattering\_raf
}
{ %%%%%% Algorithm summary %%%%%%
Calculation of sample area for scattering probes such as the FSSP, CAS, etc.
}
{ %%%%%% Category %%%%%%
Microphysics
}
{ %%%%%% Inputs %%%%%%
DOF & Coeff. & Depth of field [m] \\
BD & Coeff. & Beam diameter [m]
}
{ %%%%%% Outputs %%%%%%
SA & Coeff. & Sample area [m$^2$]
}
{ %%%%%% Formula %%%%%%
\begin{displaymath}
SA = (DOF) (BD)
\end{displaymath}
}
{ %%%%%% Author %%%%%%
NCAR-RAF
}
{ %%%%%% References %%%%%% 
NCAR-RAF Bulletin No. 24 -- \href{http://www.eol.ucar.edu/raf/Bulletins/bulletin24.html}{http://www.eol.ucar.edu/raf/Bulletins/bulletin24.html}
}



%% $Date: 2010-09-17 17:53:24 +0200 (Fri, 17 Sep 2010) $
%% $Revision: 24 $
\index{sample\_volume\_general\_raf}
\algdesc{Sample Volume}
{ %%%%%% Algorithm name %%%%%%
sample\_volume\_general\_raf
}
{ %%%%%% Algorithm summary %%%%%%
Calculates sample volume for microphysics probes (1D, 2D, FSSP, etc).
}
{ %%%%%% Category %%%%%%
Microphysics
}
{ %%%%%% Inputs %%%%%%
$V_t$ & Vector & True air speed [m/s] \\
SA & Coeff. & Sample area of probe [m$^2$] \\
$T_s$ & Coeff. & Sample rate [s] \\
}
{ %%%%%% Outputs %%%%%%
SV & Vector & Sample volume [m$^3$]
}
{ %%%%%% Formula %%%%%%
\begin{displaymath}
 SV = V_t T_s SA
\end{displaymath}
}
{ %%%%%% Author %%%%%%
NCAR-RAF
}
{ %%%%%% References %%%%%% 
NCAR-RAF Bulletin No. 24 -- \href{http://www.eol.ucar.edu/raf/Bulletins/bulletin24.html}{http://www.eol.ucar.edu/raf/Bulletins/bulletin24.html}
}






\chapter{Biophysics}

%% $Date: 2010-09-22 14:13:50 +0200 (Mi, 22. Sep 2010) $
%% $Revision: 26 $
\index{NDVI}
\algdescgeneral{NDVI}
%
{ %%%%%% Algorithm name %%%%%%
biophys\_indices (NDVI is one index calculated within the overall program)
}
%
{ %%%%%% Algorithm summary %%%%%%
Calculation of Normalised Difference Vegetation index (NDVI)
}
%
{ %%%%%% Category %%%%%%
Biophysics - broad band VIS
}
%
{ %%%%%% Inputs %%%%%%
Multi- or hyperspectral imagery (ENVI standard image data) including channels close to the wavelengths of 671nm and 864nm.\bigskip
}
%
{ %%%%%% Outputs %%%%%%
Single band with NDVI values
}
%
{ %%%%%% Formula %%%%%%
\begin{displaymath}
 \frac{R_{864}-R_{671}}{R_{864}+R_{671}}
\end{displaymath}
}
%
{ %%%%%% Author %%%%%%
DLR-DFD
}
%
{ %%%%%% References %%%%%% 
Rouse, J. W., Haas, R. H., Schell, J. A. and Deering, J. A. (1973). Monitoring vegetation systems in the great plains with erts. In: Proceedings of the Third Symposium on Significant Results Obtained with ERTS  Vol. 1, p. 309–317
}




\include{algorithms/biophysics/RVI}

\include{algorithms/biophysics/MCARI}

%% $Date: 2010-09-22 14:13:50 +0200 (Mi, 22. Sep 2010) $
%% $Revision: 26 $
\index{LCI}
\algdescgeneral{LCI}
{ %%%%%% Algorithm name %%%%%%
biophys\_indices (LCI is one index calculated within the overall program)
}
{ %%%%%% Algorithm summary %%%%%%
Calculation of Leaf Chlorophyll Index (LCI)
}
{ %%%%%% Category %%%%%%
Biophysics - narrow band chlorophyll indices
}
{ %%%%%% Inputs %%%%%%
Multi- or hyperspectral imagery (ENVI standard image data) including channels close to the wavelengths of 710nm and 850nm.\bigskip
}
{ %%%%%% Outputs %%%%%%
Single band with LCI values
}
{ %%%%%% Formula %%%%%%
\begin{displaymath}
LCI = \frac{R_{850}-R_{710}}{R_{850}+R_{710}}
\end{displaymath}
}
{ %%%%%% Author %%%%%%
DLR-DFD
}
{ %%%%%% References %%%%%% 

}




\include{algorithms/biophysics/SR705}

%% $Revision: 26 $
\index{mND705}
\algdescgeneral{mND705}
%
{ %%%%%% Algorithm name %%%%%%
biophys\_indices (PRI is one index calculated within the overall program)
}
%
{ %%%%%% Algorithm summary %%%%%%
Calculation of Chlorophyll-Index mND705 // hyperbolic regression
}
%
{ %%%%%% Category %%%%%%
Biophysics - narrow band chlorophyll indices
}
%
{ %%%%%% Inputs %%%%%%
Narrow band multi- or hyperspectral imagery (ENVI standard image data) including channels close to the wavelengths of 445nm, 705nm and 750nm.\bigskip
}
%
{ %%%%%% Outputs %%%%%%
Single band with PRI values
}
%
{ %%%%%% Formula %%%%%%
\begin{displaymath}
mND705=\frac{\rho _{750} - \rho _{705}}{\rho _{750}+\rho _{705} - 2*\rho _{445}}
\end{displaymath}
}
%
{ %%%%%% Author %%%%%%
DLR-DFD
}
%
{ %%%%%% References %%%%%% 
Sims, D.A., Gamon, J.A. (2002): Relationships between leaf pigment content and spectral reflectance across a wide range of species, leaf structures and developmental stages. In: Remote Sensing of Environment, 81, p.337-354.
}


\include{algorithms/biophysics/GI}

\include{algorithms/biophysics/PRI}

\include{algorithms/biophysics/REIP}

\include{algorithms/biophysics/DGVI1}

\include{algorithms/biophysics/DGVI2}

\include{algorithms/biophysics/NDNI}

\include{algorithms/biophysics/NDLI}

\include{algorithms/biophysics/CAI}

\include{algorithms/biophysics/CSI2}

\include{algorithms/biophysics/NDWI}

\include{algorithms/biophysics/NDWI_MIR}

\include{algorithms/biophysics/LWVI1}

\include{algorithms/biophysics/LWVI2}

\include{algorithms/biophysics/DWSI5}

\include{algorithms/biophysics/SWIRVI}

\include{algorithms/biophysics/SWIRLI}

\include{algorithms/biophysics/SWIRSI}

\include{algorithms/biophysics/clay_1}

\include{algorithms/biophysics/iron_1}




\chapter{Quality Control}

\include{algorithms/quality_control/nav_chk}

\include{algorithms/quality_control/nav_const}

\printindex

%%%%%%%%%%%%%%%%%%%%%%%%%%%%%%%%%%%%%%%%%%%%%%%%%%%%%%%%%%%%%%%%%%%%%%%%%%%%%%%%%%%%%%%%%%%%%%%%%%
%%%%%%%%%%% APPENDIX inclusions
%%%%%%%%%%%%%%%%%%%%%%%%%%%%%%%%%%%%%%%%%%%%%%%%%%%%%%%%%%%%%%%%%%%%%%%%%%%%%%%%%%%%%%%%%%%%%%%%%%
%\begin{appendix}
%\include{02_appendices/01/appendix_01}
%\end{appendix}



%%%%%%%%%%%%%%%%%%%%%%%%%%%%%%%%%%%%%%%%%%%%%%%%%%%%%%%%%%%%%%%%%%%%%%%%%%%%%%%%%%%%%%%%%%%%%%%%%%
%%%%%%%%%%% Bibliography
%%%%%%%%%%%%%%%%%%%%%%%%%%%%%%%%%%%%%%%%%%%%%%%%%%%%%%%%%%%%%%%%%%%%%%%%%%%%%%%%%%%%%%%%%%%%%%%%%%
%\newpage\addcontentsline{toc}{section}{Bibliography and references}
%\begin{thebibliography}{5}
%\bibitem{Vahdat et al., 2002} Vahdat A., Yocum K., Walsh K., Mahadevan P., Kosti\c{c} D., Chase J., and Becker D. (2002). Scalability and Accuracy in a Large-Scale Network Emulator. \emph{Proceedings of 5th Symposium on Operating Systems Design and Implementation (OSDI)}
%\bibitem{Modelnet} Modelnet. http://issg.cs.duke.edu/modelnet.html.
%\bibitem{ModelNetRelease} Modelnet release page. http://sysnet.ecsd.edu/modelnet/realease.html.
%\bibitem{ModelNetHowto} Modelnet Howto. http://sysnet.ucsd.edu/modelnet/howto.html.
%\bibitem{Holenstein, 2005} Holenstein R. (2005). Scalability and Accuracy in a Large-Scale Network Emulator - CPSC 538A - Paper Presentation
%\bibitem{Harpin, 2001} Harpin, T. (2001). Using java.lang.reflect.Proxy to Interpose on Java Class Methods. \emph{Sun Developer Network (SDN)}. http://java.sun.com/developer/technicalArticles/JavaLP/Interposing/.
%\bibitem{Kostic, 2006} Kostic D. (2005). Modelnet notes. http://resolute.ucsd.edu/dokuwiki/notes:modelnet.
%\end{thebibliography}


\end{document}
%%%%%%%%%%%%%%%%%%%%%%%%%%%%%%%%%%%%%%%%%%%%%%%%%%%%%%%%%%%%%%%%%%%%%%%%%%%%%%%%%%%%%%%%%%%%%%%%%%
%%%%%%%%%%% END OF FILE
%%%%%%%%%%%%%%%%%%%%%%%%%%%%%%%%%%%%%%%%%%%%%%%%%%%%%%%%%%%%%%%%%%%%%%%%%%%%%%%%%%%%%%%%%%%%%%%%%%
